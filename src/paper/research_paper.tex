 \documentclass[11pt, a4paper, leqno]{article}
\usepackage{a4wide}
\usepackage[T1]{fontenc}
\usepackage[utf8]{inputenc}
\usepackage{float, afterpage, rotating, graphicx}
\usepackage{epstopdf}
\usepackage{longtable, booktabs, tabularx}
\usepackage{fancyvrb, moreverb, relsize}
\usepackage{eurosym, calc}
% \usepackage{chngcntr}
\usepackage{amsmath, amssymb, amsfonts, amsthm, bm}
\usepackage{caption}
\usepackage{mdwlist}
\usepackage{xfrac}
\usepackage{setspace}
\usepackage{xcolor}
\usepackage{subcaption}
\usepackage{minibox}
% \usepackage{pdf14} % Enable for Manuscriptcentral -- can't handle pdf 1.5
% \usepackage{endfloat} % Enable to move tables / figures to the end. Useful for some submissions.


\usepackage[
    natbib=true,
    bibencoding=inputenc,
    bibstyle=authoryear-ibid,
    citestyle=authoryear-comp,
    maxcitenames=3,
    maxbibnames=10,
    useprefix=false,
    sortcites=true,
    backend=biber
]{biblatex}
\AtBeginDocument{\toggletrue{blx@useprefix}}
\AtBeginBibliography{\togglefalse{blx@useprefix}}
\setlength{\bibitemsep}{1.5ex}
\addbibresource{refs.bib}





\usepackage[unicode=true]{hyperref}
\hypersetup{
    colorlinks=true,
    linkcolor=black,
    anchorcolor=black,
    citecolor=black,
    filecolor=black,
    menucolor=black,
    runcolor=black,
    urlcolor=black
}


\widowpenalty=10000
\clubpenalty=10000

\setlength{\parskip}{1ex}
\setlength{\parindent}{0ex}
\setstretch{1.5}


\begin{document}

\title{effectiveness of incentives for research and development\thanks{Solmaz Ahmadi, University of Bonn. Email: \href{mailto:s6soahma@uni-bonn.de}{\nolinkurl{s6soahma [at] uni-bonn [dot] de}}.}}

\author{Solmaz Ahmadi}

\date{
    {\bf }
    \\[1ex]
    \today
}

\maketitle


\begin{abstract}
    This paper investigates the effectiveness of a unique R&D subsidy program implemented in northern Italy.
    The government asked firms to submit proposals for new projects, and among them, only projects with a
    score above a certain threshold received the subsidy. They use a sharp regression discontinuity design
    to compare the investment spending of subsidized firms with that of unsubsidized firms. They observe
    evidence that for the whole sample there is no significant increase in investment. However, removing the
    heterogeneity in the size of firms, their findings show public funds increase small firms' investments,
    whereas larger firms were not affected by subsidy.
\end{abstract}
\clearpage

\section{Introduction} % (fold)
\label{sec:introduction}
Bronzini and Lachini (2014) study the effects of a unique R\&D subsidy program
executed in northern Italy on investment expenditures of firms.
The public R\&D funding is a government policy that aims to trigger marginal projects,
those that would not be carried out without the subventions. The economic rationale
behind the R\&D subsidies is firstly to conquer market failure of knowledge as a
public good the positive externalities of which cannot be fully internalized by the
firm. Secondly, the government grants aid to overcome the firm's liquidity
constraints. Firms were asked to present a proposal for their new projects and
an independent technical committee scores them. Only firms whose scores were exceeded
a specific threshold were awarded public grants. To estimate the causal impact of
subsidies, Bronzini and Lachini (2014) apply a sharp regression discontinuity design
(RDD) comparing the private investment spending of funded and nonfunded firms with
scores close to the threshold. Among nonexperimental econometric methods, the
regression discontinuity controls preferably for the endogeneity of treatment since
it can be shown as a randomized experiment by arguing that the agents had been
randomly drawn just below or just above the cutoff. The paper finds for the whole
sample of firms, the investment expenditures do not increase significantly.
since the overall impact hides the considerable heterogeneity in the program’s effect,
Bronzini and Lachini (2014) divide the sample by small and large firms and demonstrate
that although the subsidy did not affect large enterprises' investment spending,
small companies raised their investments by roughly the amount of the grant they
gained.


If you are using this template, please cite this item from the references: \citet{GaudeckerEconProjectTemplates}

\citet{Schelling69} example in the code is taken from \citet{StachurskiSargent13}

The decision rule of an agent is the following:
\begin{align*}
    \text{move} & \quad \text{if} \quad n_\text{neighbours} < 4 \\
    \text{stay} & \quad \text{if} \quad n_\text{neighbours} \geq 4
\end{align*}


\begin{pdf}
    \caption{Segregation by cycle in the baseline \citet{Schelling69} model as in the \citet{StachurskiSargent13} example}
    \includegraphics[width=\textwidth]{output.txt}

    \includegraphics[width=\textwidth]{../../bld/figures/baseline/regression_degree_0}

    \caption{Segregation by cycle in the baseline \citet{Schelling69} model as in the \citet{StachurskiSargent13} example}

    \includegraphics[width=\textwidth]{../../bld/figures/baseline/regression_degree_1}

    \caption{Segregation by cycle in the baseline \citet{Schelling69} model as in the \citet{StachurskiSargent13} example}

    \includegraphics[width=\textwidth]{../../bld/figures/baseline/regression_degree_2}

    \caption{Segregation by cycle in the baseline \citet{Schelling69} model as in the \citet{StachurskiSargent13} example}

    \includegraphics[width=\textwidth]{../../bld/figures/baseline/regression_degree_3}

\end{pdf}


\begin{figure}
    \caption{Segregation by cycle in the baseline \citet{Schelling69} model, limiting the number of potential moves per period to two}

    \includegraphics[width=\textwidth]{../../bld/figures/schelling_max_moves_2}

\end{figure}

% section introduction (end)




\setstretch{1}
\printbibliography
\setstretch{1.5}




% \appendix

% The chngctr package is needed for the following lines.
% \counterwithin{table}{section}
% \counterwithin{figure}{section}

\end{document}
